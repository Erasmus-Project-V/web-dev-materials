% --- LaTeX Lecture Notes Template - S. Venkatraman ---

% --- Set document class and font size ---

\documentclass[letterpaper, 12pt]{article}

% --- Package imports ---

% Extended set of colors
\usepackage[dvipsnames]{xcolor}

\usepackage{
  amsmath, amsthm, amssymb, mathtools, dsfont, units,          % Math typesetting
  graphicx, wrapfig, subfig, float,                            % Figures and graphics formatting
  listings, color, inconsolata, pythonhighlight,               % Code formatting
  fancyhdr, sectsty, hyperref, enumerate, enumitem, framed }   % Headers/footers, section fonts, links, lists

\usepackage{listings}

\lstset{ %
    backgroundcolor=\color{white},   % choose the background color; you must add \usepackage{color} or \usepackage{xcolor}
    basicstyle=\footnotesize\ttfamily,        % the size of the fonts that are used for the code
    breakatwhitespace=false,         % sets if automatic breaks should only happen at whitespace
    breaklines=true,                 % sets automatic line breaking
    captionpos=b,                    % sets the caption-position to bottom
    commentstyle=\color{mygreen},    % comment style
    deletekeywords={...},            % if you want to delete keywords from the given language
    escapeinside={\%*}{*)},          % if you want to add LaTeX within your code
extendedchars=true,              % lets you use non-ASCII characters; for 8-bits encodings only, does not work with UTF-8
frame=single,                    % adds a frame around the code
keepspaces=true,                 % keeps spaces in text, useful for keeping indentation of code (possibly needs columns=flexible)
keywordstyle=\color{blue},       % keyword style
otherkeywords={*,...},            % if you want to add more keywords to the set
numbers=left,                    % where to put the line-numbers; possible values are (none, left, right)
numbersep=5pt,                   % how far the line-numbers are from the code
numberstyle=\tiny\color{mygray}, % the style that is used for the line-numbers
rulecolor=\color{black},         % if not set, the frame-color may be changed on line-breaks within not-black text (e.g. comments (green here))
showspaces=false,                % show spaces everywhere adding particular underscores; it overrides 'showstringspaces'
showstringspaces=false,          % underline spaces within strings only
showtabs=false,                  % show tabs within strings adding particular underscores
stepnumber=2,                    % the step between two line-numbers. If it's 1, each line will be numbered
stringstyle=\color{mymauve},     % string literal style
tabsize=2,                       % sets default tabsize to 2 spaces % show the filename of files included with \lstinputlisting; also try caption instead of title,
}

\renewcommand{\lstlistingname}{Popis}

% lipsum is just for generating placeholder text and can be removed
\usepackage{hyperref, lipsum} 

% --- Fonts ---

\usepackage{newpxtext, newpxmath, inconsolata}

% --- Page layout settings ---

% Set page margins
\usepackage[left=1.35in, right=1.35in, top=1.0in, bottom=.9in, headsep=.2in, footskip=0.35in]{geometry}

% Anchor footnotes to the bottom of the page
\usepackage[bottom]{footmisc}

\usepackage{amsmath, amsfonts, mathtools, amsthm, amssymb}
\usepackage{mathrsfs}

% Set line spacing
\renewcommand{\baselinestretch}{1.2}

% Set spacing between paragraphs
\setlength{\parskip}{1.3mm}

% Allow multi-line equations to break onto the next page
\allowdisplaybreaks

% --- Page formatting settings ---

% Set image captions to be italicized
\usepackage[font={it,footnotesize}]{caption}

% Set link colors for labeled items (blue), citations (red), URLs (orange)
\hypersetup{colorlinks=true, linkcolor=RoyalBlue, citecolor=RedOrange, urlcolor=ForestGreen}

% Set font size for section titles (\large) and subtitles (\normalsize) 
\usepackage{titlesec}
\titleformat{\section}{\large\bfseries}{{\fontsize{19}{19}\selectfont\textreferencemark}\;\; }{0em}{}
\titleformat{\subsection}{\normalsize\bfseries\selectfont}{\thesubsection\;\;\;}{0em}{}

% Enumerated/bulleted lists: make numbers/bullets flush left
%\setlist[enumerate]{wide=2pt, leftmargin=16pt, labelwidth=0pt}
\setlist[itemize]{wide=0pt, leftmargin=16pt, labelwidth=10pt, align=left}

% --- Table of contents settings ---
\usepackage[subfigure]{tocloft}
\usepackage{graphics}
\usepackage{graphicx}

% Reduce spacing between sections in table of contents
\setlength{\cftbeforesecskip}{.9ex}

% Remove indentation for sections
\cftsetindents{section}{0em}{0em}

% Set font size (\large) for table of contents title
\renewcommand{\cfttoctitlefont}{\large\bfseries}

% Remove numbers/bullets from section titles in table of contents
\makeatletter
\renewcommand{\cftsecpresnum}{\begin{lrbox}{\@tempboxa}}
\renewcommand{\cftsecaftersnum}{\end{lrbox}}
\makeatother

\lstdefinelanguage{JavaScript}{
keywords={typeof, new, true, false, catch, function, return, null, catch, switch, var, if, in, while, do, else, case, break},
keywordstyle=\color{blue}\bfseries,
ndkeywords={class, export, boolean, throw, implements, import, this},
ndkeywordstyle=\color{darkgray}\bfseries,
identifierstyle=\color{black},
sensitive=false,
comment=[l]{//},
morecomment=[s]{/*}{*/},
commentstyle=\color{purple}\ttfamily,
stringstyle=\color{red}\ttfamily,
morestring=[b]',
morestring=[b]"
}

\lstdefinelanguage{CSS}{
keywords={accelerator,azimuth,background,background-attachment,
background-color,background-image,background-position,
background-position-x,background-position-y,background-repeat,
behavior,border,border-bottom,border-bottom-color,
border-bottom-style,border-bottom-width,border-collapse,
border-color,border-left,border-left-color,border-left-style,
border-left-width,border-right,border-right-color,
border-right-style,border-right-width,border-spacing,
border-style,border-top,border-top-color,border-top-style,
border-top-width,border-width,bottom,caption-side,clear,
clip,color,content,counter-increment,counter-reset,cue,
cue-after,cue-before,cursor,direction,display,elevation,
empty-cells,filter,float,font,font-family,font-size,
font-size-adjust,font-stretch,font-style,font-variant,
font-weight,height,ime-mode,include-source,
layer-background-color,layer-background-image,layout-flow,
layout-grid,layout-grid-char,layout-grid-char-spacing,
layout-grid-line,layout-grid-mode,layout-grid-type,left,
letter-spacing,line-break,line-height,list-style,
list-style-image,list-style-position,list-style-type,margin,
margin-bottom,margin-left,margin-right,margin-top,
marker-offset,marks,max-height,max-width,min-height,
min-width,-moz-binding,-moz-border-radius,
-moz-border-radius-topleft,-moz-border-radius-topright,
-moz-border-radius-bottomright,-moz-border-radius-bottomleft,
-moz-border-top-colors,-moz-border-right-colors,
-moz-border-bottom-colors,-moz-border-left-colors,-moz-opacity,
-moz-outline,-moz-outline-color,-moz-outline-style,
-moz-outline-width,-moz-user-focus,-moz-user-input,
-moz-user-modify,-moz-user-select,orphans,outline,
outline-color,outline-style,outline-width,overflow,
overflow-X,overflow-Y,padding,padding-bottom,padding-left,
padding-right,padding-top,page,page-break-after,
page-break-before,page-break-inside,pause,pause-after,
pause-before,pitch,pitch-range,play-during,position,quotes,
-replace,richness,right,ruby-align,ruby-overhang,
ruby-position,-set-link-source,size,speak,speak-header,
speak-numeral,speak-punctuation,speech-rate,stress,
scrollbar-arrow-color,scrollbar-base-color,
scrollbar-dark-shadow-color,scrollbar-face-color,
scrollbar-highlight-color,scrollbar-shadow-color,
scrollbar-3d-light-color,scrollbar-track-color,table-layout,
text-align,text-align-last,text-decoration,text-indent,
text-justify,text-overflow,text-shadow,text-transform,
text-autospace,text-kashida-space,text-underline-position,top,
unicode-bidi,-use-link-source,vertical-align,visibility,
voice-family,volume,white-space,widows,width,word-break,
word-spacing,word-wrap,writing-mode,z-index,zoom},
sensitive=true,
morecomment=[l]{//},
morecomment=[s]{/*}{*/},
morestring=[b]',
morestring=[b]",
alsoletter={:},
alsodigit={-}
}

% --- Set path for images ---

\graphicspath{{images/}{../images/}}

% --- Math/Statistics commands ---

% Add a reference number to a single line of a multi-line equation
% Usage: "\numberthis\label{labelNameHere}" in an align or gather environment
\newcommand\numberthis{\addtocounter{equation}{1}\tag{\theequation}}

% Shortcut for bold text in math mode, e.g. $\b{X}$
\let\b\mathbf

% Shortcut for bold Greek letters, e.g. $\bg{\beta}$
\let\bg\boldsymbol

% Shortcut for calligraphic script, e.g. %\mc{M}$
\let\mc\mathcal

% \mathscr{(letter here)} is sometimes used to denote vector spaces
\usepackage[mathscr]{euscript}
\usepackage[croatian]{babel}
% Convergence: right arrow with optional text on top
% E.g. $\converge[p]$ for converges in probability
\newcommand{\converge}[1][]{\xrightarrow{#1}}

% Weak convergence: harpoon symbol with optional text on top
% E.g. $\wconverge[n\to\infty]$
\newcommand{\wconverge}[1][]{\stackrel{#1}{\rightharpoonup}}

% Equality: equals sign with optional text on top
% E.g. $X \equals[d] Y$ for equality in distribution
\newcommand{\equals}[1][]{\stackrel{\smash{#1}}{=}}

% Normal distribution: arguments are the mean and variance
% E.g. $\normal{\mu}{\sigma}$
\newcommand{\normal}[2]{\mathcal{N}\left(#1,#2\right)}

% Uniform distribution: arguments are the left and right endpoints
% E.g. $\unif{0}{1}$
\newcommand{\unif}[2]{\text{Uniform}(#1,#2)}

% Independent and identically distributed random variables
% E.g. $ X_1,...,X_n \iid \normal{0}{1}$
\newcommand{\iid}{\stackrel{\smash{\text{iid}}}{\sim}}

% Sequences (this shortcut is mostly to reduce finger strain for small hands)
% E.g. to write $\{A_n\}_{n\geq 1}$, do $\bk{A_n}{n\geq 1}$
\newcommand{\bk}[2]{\{#1\}_{#2}}

% Math mode symbols for common sets and spaces. Example usage: $\R$
\newcommand{\R}{\mathbb{R}}	% Real numbers
\newcommand{\C}{\mathbb{C}}	% Complex numbers
\newcommand{\Q}{\mathbb{Q}}	% Rational numbers
\newcommand{\Z}{\mathbb{Z}}	% Integers
\newcommand{\N}{\mathbb{N}}	% Natural numbers
\newcommand{\F}{\mathcal{F}}	% Calligraphic F for a sigma algebra
\newcommand{\El}{\mathcal{L}}	% Calligraphic L, e.g. for L^p spaces

% Math mode symbols for probability
\newcommand{\pr}{\mathbb{P}}	% Probability measure
\newcommand{\E}{\mathbb{E}}	% Expectation, e.g. $\E(X)$
\newcommand{\var}{\text{Var}}	% Variance, e.g. $\var(X)$
\newcommand{\cov}{\text{Cov}}	% Covariance, e.g. $\cov(X,Y)$
\newcommand{\corr}{\text{Corr}}	% Correlation, e.g. $\corr(X,Y)$
\newcommand{\B}{\mathcal{B}}	% Borel sigma-algebra

% Other miscellaneous symbols
\newcommand{\tth}{\text{th}}	% Non-italicized 'th', e.g. $n^\tth$
\newcommand{\Oh}{\mathcal{O}}	% Big-O notation, e.g. $\O(n)$
\newcommand{\1}{\mathds{1}}	% Indicator function, e.g. $\1_A$

% Additional commands for math mode
\DeclareMathOperator*{\argmax}{argmax}		% Argmax, e.g. $\argmax_{x\in[0,1]} f(x)$
\DeclareMathOperator*{\argmin}{argmin}		% Argmin, e.g. $\argmin_{x\in[0,1]} f(x)$
\DeclareMathOperator*{\spann}{Span}		% Span, e.g. $\spann\{X_1,...,X_n\}$
\DeclareMathOperator*{\bias}{Bias}		% Bias, e.g. $\bias(\hat\theta)$
\DeclareMathOperator*{\ran}{ran}			% Range of an operator, e.g. $\ran(T) 
\DeclareMathOperator*{\dv}{d\!}			% Non-italicized 'with respect to', e.g. $\int f(x) \dv x$
\DeclareMathOperator*{\diag}{diag}		% Diagonal of a matrix, e.g. $\diag(M)$
\DeclareMathOperator*{\trace}{trace}		% Trace of a matrix, e.g. $\trace(M)$
\DeclareMathOperator*{\supp}{supp}		% Support of a function, e.g., $\supp(f)$

% Numbered theorem, lemma, etc. settings - e.g., a definition, lemma, and theorem appearing in that 
% order in Lecture 2 will be numbered Definition 2.1, Lemma 2.2, Theorem 2.3. 
% Example usage: \begin{theorem}[Name of theorem] Theorem statement \end{theorem}
\theoremstyle{definition}
\newtheorem{theorem}{Theorem}[section]
\newtheorem{proposition}[theorem]{Proposition}
\newtheorem{lemma}[theorem]{Lemma}
\newtheorem{corollary}[theorem]{Corollary}
\newtheorem{definition}[theorem]{Definicija}
\newtheorem{example}[theorem]{Example}
\newtheorem{remark}[theorem]{Remark}


% Un-numbered theorem, lemma, etc. settings
% Example usage: \begin{lemma*}[Name of lemma] Lemma statement \end{lemma*}
\newtheorem*{theorem*}{Theorem}
\newtheorem*{proposition*}{Proposition}
\newtheorem*{lemma*}{Lemma}
\newtheorem*{corollary*}{Corollary}
\newtheorem*{definition*}{Definition}
\newtheorem*{example*}{Example}
\newtheorem*{remark*}{Remark}
\newtheorem*{claim}{Claim}


%
%\declaretheorem[numbered=no, style=thmexplanationbox, name=Proof]{explanation}
%\declaretheorem[numbered=no, style=thmproofbox, name=Proof]{replacementproof}
%\declaretheorem[style=thmbluebox,  numbered=no, name=Exercise]{ex}
%\declaretheorem[style=thmbluebox,  numbered=no, name=Example]{eg}
%\declaretheorem[style=thmblueline, numbered=no, name=Remark]{remark}
%\declaretheorem[style=thmblueline, numbered=no, name=Note]{note}
%
%\renewenvironment{proof}[1][\proofname]{\begin{replacementproof}}{\end{replacementproof}}
%
%\AtEndEnvironment{eg}{\null\hfill$\diamond$}%
%
%\newtheorem*{uovt}{UOVT}
%\newtheorem*{notation}{Notation}
%\newtheorem*{previouslyseen}{As previously seen}
%\newtheorem*{problem}{Problem}
%\newtheorem*{observe}{Observe}
%\newtheorem*{property}{Property}
%\newtheorem*{intuition}{Intuition}
%
%
%\usepackage{etoolbox}
%\AtEndEnvironment{vb}{\null\hfill$\diamond$}%
%\AtEndEnvironment{intermezzo}{\null\hfill$\diamond$}%

% --- Left/right header text (to appear on every page) ---

% Do not include a line under header or above footer
\pagestyle{fancy}
\renewcommand{\footrulewidth}{0pt}
\renewcommand{\headrulewidth}{0pt}

% Right header text: Lecture number and title
\renewcommand{\sectionmark}[1]{\markright{#1} }
\fancyhead[R]{\small\textit{\nouppercase{\rightmark}}}

% Left header text: Short course title, hyperlinked to table of contents
\fancyhead[L]{\hyperref[sec:contents]{\small Uvod u mreže}}

% --- Document starts here ---

\begin{document}

% --- Main title and subtitle ---

\title{\Large{\textbf{\MakeUppercase{Osnove web programiranja}}} \\[1em]
\normalsize \textbf{Uvod u Internet i računalne mreže}}

% --- Author and date of last update ---

\author{\normalsize Karlo Vizec}
\date{\normalsize\vspace{-1ex} Zadnji put ažurirano: \today}

% --- Add title and table of contents ---

\maketitle
\tableofcontents\label{sec:contents}

% --- Main content: import lectures as subfiles ---

% TeX root = ../main.tex


% First argument to \section is the title that will go in the table of contents. Second argument is the title that will be printed on the page.
\section[Računalne mreže]{Računalne mreže}

U početku su računala bila ogromni strojevi, zuzimali su cijele sobe i bili su sporiji od najsporijeg pametnog mobitela.
I što je najbitnije, nisu bila povezana.
Jedno računalo je bilo samo za sebe, nije imalo pristup ostalima, a to je znatno ograničavalo njihovu funkcinonalnost.
Podaci su morali biti prebacivani između računala pomoću fizičkih uređaja, diskete, a to je potrajalo i nije bilo praktično za velike udaljenosti.

Informatičari su to ubrzo shvatili te nije prošlo mnogo vremena od razvoja prvih računala do razvoja prve veće računalne mreže.
Ta je mreža bila \textbf{ARPANET}.
Razvila ju je američka agencija ARPA, sada zvana DARPA, 1969.
Do 1971.\ je imala 15 računala povezana na nju.
Za razliku od današnjeg Interneta, ARPANET bio je znatno manje sofisticiran, sigurnosne mjere nisu postojale, umreženi računala bilo je malo i bio je spor.
Ključna razlika je činjenica da su ARPANET i slične mreže koristili većinom informatički stručnjaci, sveučilišta i vlada, a ne i svakodnevni građani.

Međutim, ARPANET je bio ključan korak u razvoju modernog Interneta.
ARPANET je bio rani primjer većeg \textbf{distribuiranog sustava}.
Distribuirani sustavi su ključni sastojak modernog računarstva jer omogoćuju podjelu rada na više računala, a to znatno povećava performanse.
Jedno računalo, makar bilo i superračunalo, nije dovoljno snažno za zadatke koji zahtjevaju veliku količinu računalnih resursa.
Očiti primjeri su znanstveni izračuni, npr.\ simulacija proteina.
Naravno, sam Internet je najveći primjer masovnog distribuiranog sustava.

\definition{Distribuirani sustav je sustav računalne mreže u koju su povezana dva ili više zasebna računala koja koriste svoje računalne resure za izvršavanje zajedničkog zadatka ili cilja.}

Danas je sve više i više uređaja umreženo.
Štoviše, svjetska ekonomija ovisi o Internetu.
Tijekom zadnjih nekoliko godina, sve je više i Internet of Things (IoT) uređaja, npr. pametni televizori, frižideri, termostati i slično.

\subsection{Osnove vrste i svrha mreža}

Internet kao koncept najlakše se može zamisliti kao \("\)mreža od mreža\("\).
Računalne mreže postoje kako bi omogućile međusobno povezivanje računala i drugih uređaja u svrhe međusobne komunikacije.

Te računalne mreže od kojih se sastoji Internet uključuju one najmanje, \textbf{LAN} (engl. \textit{Local Area Network}) mreže koje se nalaze u kućama i stanovima svih ljudi.

\definition{LAN je vrsta lokalne mreže koja povezuje uređaje na lokalnom prostoru, npr. kućanstvo, sveučilište ili tvrtka.}

One se obično sastoje od rutera, nekoliko računala i mobitela, televizora te još neke opreme.
Sljedeća najbitnija razina mreže je \textbf{WAN} (engl. \textit{Wide Area Network}).
Glavna razlika između WAN i LAN mreža je u geografskom području koje zauzimaju.
WAN mreže će obično imati naprednije sustave koji su potrebni za upravljanje tako velikim prostorom.

\definition{WAN je vrsta mreže koja se sastoji od velikog broja uređaja, najčešće i druge LAN mreže, s velikim opsegom, npr. grad ili država.}

Još kategorija računalnih mreža postoje, uključujući \textbf{Personal Area Network (PAN)} i \textbf{Metropolitan Area Network (MAN)}.
PAN mreže su nekoliko metara u veličini.
Primjer takve mreže su mobitel i slušalice spojene pomoću WiFi-a ili Bluetootha.
MAN mreže obično su one mreže koje povezuju veće područje, recimo neki grad ili naselje.

Pristup Internetu daju pružatelji internetskih usluga (engl. \textit{Internet Service Provider}, \textbf{ISP}).
Primjeri ISP-ova u Hrvatskoj su Hrvatski Telekom, Optima Telekom, A1 i drugi.

\definition{ISP je organizacija (u većini slučajeva komercijalna tvrtka) koja pruža internetske usluge drugim osobama i organizacijama.}

\subsection{Komponente mreža}

Za veliku većinu osoba, pristup Internetu dobiti će u obliku uređaja koji se zove \textbf{ruter} (od engl. \textit{router}).
Ruter je uređaj koji služi kao glavna pristupna točka Internetu, na njega se, žično ili bežično, povezuju drugi uređaji koji zahtjevaju pristup Internetu, ali i ostalim uređajima u LAN mreži.
Njegova svrha je komunikacija između uređaja u njegovoj mreži i šireg Interneta.
On odašilje i prima (te onda šalje uređaju koji je namijenjeni primatelj) pakete podataka koje šalju i primaju uređaji povezani u mrežu putem rutera.

\definition{Ruter je uređaj koji povezuje računalne mreže te računala u njima tako što koordinirate slanje i primanje paketa podataka. Najčešće povezuje lokalnu mrežu s Internetom.}

\begin{figure}[H]\label{fig:lan-diagram}
\centering
\vspace*{-1cm}
\includegraphics[scale=0.1]{lan-diagram}
\vspace*{-1cm}
\caption{Dijagram jednostavne LAN mreže s ruterom i nekoliko uređaja povezanih na njega. Kompliciranije LAN mreže mogu imati i mrežni preklopnik (engl. \textit{switch}). Izvor: Cloudflare.}
\end{figure}

Komponente mreže prikazane na Slici 1.1 jesu one najčešće.
Međutim, postoje i druge komponente mreža.
Ruter, recimo, nije nužan da bi se stvorila mreža.
Primjer uređaja koji može zamjeniti neke svrhe rutera je \textbf{mrežni preklopnik}.
On povezuje računala u mrežu u obliku zvijezde tako što koordinira slanje poruka između računala i omogućava im da komuniciraju međusobno.
Glavna razlika između preklopnika i rutera je ta što preklopnik ne može sam povezati svoju mrežu s vanjskim svijetom Interneta.
Međutim, preklopnici su često spojeni na ruter te tako mogu povezati računala s ostatkom Interneta.

\definition{Mrežni preklopnik je uređaj koji povezuje računala unutar mreže tako što šalje podatkovne pakete na njihovo odredište.}

\begin{figure}[H]\label{fig:switch-lan-diagram}
\includegraphics[scale=0.3,left]{switch-diagram}
\raggedleft
\caption{Dijagram LAN mreže s mrežnim preklopnikom. Izvor: Cloudflare.}
\end{figure}
% TeX root = ../main.tex

\section[Internet]{Internet}

\subsection{Arhitektura}

Predače Interneta kao što su ARPANET bile su u početku eksluzivne.
Broj korisnika bio je malen, ograničen na profesijonalne korisnike, uglavnom sveučilišta i vladine organizacije.
Stoga je i broj \textbf{čvorova} bio relativno mali.

Za razliku od ARPANET-a, na Internet su povezane milijarde uređaja, od osobnih računala, servera do pametnih automobila.
Očekuje da će broj ne-IoT uređaja povezanih na Internet (računala, serveri, itd.) u 2025.\ dostići 10 milijardi.
Sam broj uređaja je ogroman, veći je od populacije Zemlje, a svakom godinom se drastično povećava.

\definition{Čvor je fizičko računalo koje je član računalne mreže.}

\begin{figure}[h]\label{fig:arpanet}
\includegraphics[scale=0.3125,left]{arpanet}
\raggedleft
\caption{Dijagram ARPANET mreže u rujnu 1974. Izvor: Britannica.}
\end{figure}

Dakle, zbog tolikog broja korisnika koje mora služiti, Internet mora biti dizajniran da podrži toliki broj korisnika.
Kao što smo prije spomenuli, Internet je masovni, distribuiran sustav.
Najlakše ga je zamisliti kao \textit{mrežu mreža}: ogromna mreža stvorena od velikog broja manjih mreža, npr. WAN, koje su same stvorene od još manjih mreža, npr. LAN.
Prostire se cijelom Zemljom, prolazi kroz stotine države i svih sedam kontinenata.

Internet načelno nema središnju organizaciju koja njime upravlja.
Malo je čudo što je moguće (na relativno lagan način) napraviti web stranicu kojoj netko može pristupiti putem Interneta iz (više-manje) bilo koje zemlje svijeta na gotovo bilo kojem uređaju s pristupom Internetu.
To je moguće zbog standardizacije koju provede međunarodne organizacije.

Jezgra Interneta sastoji se od nekoliko ogromnih, medunarodnih korporacija (ISP-ova) koji se vlasnici nekoliko skupina međusobno povezanih  mreža s velikim protokom.
Glavni dio tih mreža je skupina podvodnih komunikacijskih kablova koji povezuju Internet izmedu kontinenata.
Te mreže su \textbf{kralježnica Interneta}.
Ti su kablovi bitni geostrateški resursi za države, ali i korporacije.
Pomoću njih obavještajne agencije mogu provoditi kibernetičku špijunažu.\footnote{U SAD-u je ta agencija NSA, a u UK-u GCHQ.}

\begin{figure}[h]
    \includegraphics[scale = 0.42]{internet-underwater-backbone}
    \caption{Međukontinentalna mreža podvodnih komunikacijskih kablova koji služe kao kralježnica Interneta iz 2015. Izvor: OpenStreetMap. }\label{fig:figure6}
\end{figure}





\subsection{Adresiranje, domene i DNS}


\subsection{Mrežni protokoli}

\definition{Mrežni protokol je (obično standardizirani) skup pravila koja omogućuju koordinaciju i komunikaciju između dva ili više programa ili računala u mreži.}

% TeX root = ../main.tex

\section[World Wide Web]{World Wide Web}

\subsection{HTTP i prijatelji}

\subsection{Programiranje na Webu}


% --- Bibliography ---

% Start a bibliography with one item.
% Citation example: "\cite{williams}".

% --- Document ends here ---

\end{document}
